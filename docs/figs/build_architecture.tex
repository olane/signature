
\tikzstyle{block} = [draw, fill=blue!20, rectangle, 
    minimum height=4em, minimum width=8em]
\tikzstyle{sum} = [draw, fill=blue!20, circle, node distance=1cm]
\tikzstyle{input} = [coordinate]
\tikzstyle{output} = [coordinate]
\tikzstyle{pinstyle} = [pin edge={to-,thin,black}]
\tikzstyle{database} = [cylinder,
				        fill=blue!20,
				        shape border rotate=90,
        				minimum height=2.0cm,
        				minimum width=1.5cm,
				        aspect=0.25,
				        draw]

\tikzset{blue dashed/.style={draw=blue!50!white, line width=1pt,
                           dash pattern=on 2pt off 2pt,
                            inner sep=4mm, rectangle, rounded corners}}

\begin{tikzpicture}[auto, node distance=3cm,>=latex']

    % We start by placing the blocks

    % Vertical column
    \node [block, name=library] (library) {Library};
    \node [block, name=reader, below of=library] (reader) {Reader};

    % Breakout bit
    \node [block, name=songadder, right of=reader, node distance=8cm, fill=gray!30] (songadder) {Song registrar};
    \node [database, name=db, below of=songadder, node distance=3cm, fill=gray!30] (db) {Database};


    % Now we connect the lines
    \draw [->] (library) -- node {Song audio} (reader);

    \draw [->] (reader) -- node {Song audio} (songadder);
 	\draw [->] (songadder) -- node (f) {Fingerprints} (db);


 	\node[draw, blue dashed, fit=(songadder) (db) (f)] {};
\end{tikzpicture}
