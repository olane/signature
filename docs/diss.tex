
\documentclass[12pt,a4paper,twoside,openright]{report}
\usepackage[pdfborder={0 0 0}]{hyperref}    % turns references into hyperlinks
\usepackage[margin=25mm]{geometry}  % adjusts page layout
\usepackage{graphicx}  % allows inclusion of PDF, PNG and JPG images
\usepackage{verbatim}
\usepackage{docmute}   % only needed to allow inclusion of proposal.tex

\raggedbottom                           % try to avoid widows and orphans
\sloppy
\clubpenalty1000%
\widowpenalty1000%

\renewcommand{\baselinestretch}{1.1}    % adjust line spacing to make
                                        % more readable


\parindent 0pt % Disable paragraph indent and add paragraph spacing
\parskip 6pt

\begin{document}

\bibliographystyle{plain}


%%%%%%%%%%%%%%%%%%%%%%%%%%%%%%%%%%%%%%%%%%%%%%%%%%%%%%%%%%%%%%%%%%%%%%%%
% Title


\pagestyle{empty}

\rightline{\LARGE \textbf{Oliver Lane}}

\vspace*{60mm}
\begin{center}
\Huge
\textbf{Audio Fingerprinting for Music Recongition} \\[5mm]
Computer Science Tripos -- Part II \\[5mm]
Trinity Hall \\[5mm]
\today  % today's date
\end{center}

%%%%%%%%%%%%%%%%%%%%%%%%%%%%%%%%%%%%%%%%%%%%%%%%%%%%%%%%%%%%%%%%%%%%%%%%%%%%%%
% Proforma, table of contents and list of figures

\pagestyle{plain}

\chapter*{Proforma}

{\large
\begin{tabular}{ll}
Name:               & \bf Oliver Lane                       \\
College:            & \bf Trinity Hall                     \\
Project Title:      & \bf Audio Fingerprinting for Music Recognition \\
Examination:        & \bf Computer Science Tripos -- Part II, July 2015  \\
Word count:         & \bf   \\
Project Originator: & Oliver Lane                    \\
Supervisor:         & Vaiva Imbrasait\'{e}                    \\ 
\end{tabular}
}


\section*{Original Aims of the Project}

In this project I aimed firstly to implement at least two audio fingerprinting algorithms, each with a corresponding matching algorithm, for the purpose of recognising clips of songs from a library in a way that is robust to noise and distortions.

Secondly, I aimed to assemble a library of songs and corresponsing test clips against which to test these algorithms.

Finally, I aimed to compare these implementations against several criteria, including size of fingerprints generated and percentage of test clips matched correctly from the assembled library, at various clip lengths and levels of noise.

\section*{Work Completed}

Two implementations for audio fingerprinting algorithms were created, with corresponsing matching algorithms. These algorithms were proposed by Avery Wang in 2003 \cite{Wang03} and Haitsma et al. in 2002 \cite{Haitsma02}. A library of songs was assembled, and corresponding test cases made by taking different length clips from the library and adding distortions.

The two implementations were tested and evaluated against the test cases and each other.

\section*{Special Difficulties}

None.

 
\newpage
\section*{Declaration}

I, Oliver Lane of Trinity Hall, being a candidate for Part II of the Computer Science Tripos, hereby declare that this dissertation and the work described in  it are my own work, unaided except as may be specified below, and that the  dissertation does not contain material that has already been used to any  substantial extent for a comparable purpose.

\bigskip
\leftline{Signed}

\medskip
\leftline{Date}

\tableofcontents

\listoffigures


%%%%%%%%%%%%%%%%%%%%%%%%%%%%%%%%%%%%%%%%%%%%%%%%%%%%%%%%%%%%%%%%%%%%%%%
%                              CHAPTERS

\pagestyle{headings}

\chapter{Introduction}


This project concerns the implementation and comparison of two algorithms for audio fingerprinting. Each algorithms consists of a fingerprinter, which generates an audio fingerprint from a given piece of audio, and a matcher, which matches a fingerprint against a precomputed database of fingerprints for a library of known songs.

Two algorithms were successfully implemented and evaluated against a library of over 700 songs. The algorithms were evaluated against each other for several different criteria, including robustness to several types of audio distortion and the size of the resulting database.


\section{Background}

An audio fingerprint is a compact representation of some audio, which summarises its content. It can be thought of as a kind of hash, where the aim is that perceptually similar pieces of audio -- that is, audio which sounds the same or very similar to the human ear -- have similar hash values.

One of the principal uses of this technology is to retrieve metadata for songs based on a short clip. For example, a user might want to identify a song's name and artist by recording a clip of it from their mobile phone. Music labels might use the technology to automatically monitor radio stations to ensure the correct song royalties are being paid.

Audio fingerprinting algorithms need to work independently to the representation of the audio signal. For instance, two instances of the same song encoded at different rates of compression will have quite different representations but are very perceptually similar, and so should have similar fingerprints. As a result, the algorithms need to be robust to distortions. 

The specific distortions an algorithm tries to be robust towards are often dependent on the intended use cases, but most algorithms try to be resistant to background noise, cropping, and compression artefacts as a minimum.

There are a number of algorithms for audio fingerprinting which have been proposed in the literature, some of which have been used extensively for commercial applications. This project implements two algorithms from the literature and attempts to compare their strengths and weaknesses.


\section{Algorithms implemented}

Two algorithms were implemented for the project. The first of these is a commercially deployed algorithm developed by Avery Wang from Shazam \cite{Wang03}. The second is an algorithm proposed by Philips researchers Haitsma et al. in 2002 \cite{Haitsma02}.


%%%%%%%%%%%%%%%%%%%%%%%%%%%%%%%%%%%%%%%%%%%%%%%%%%%%%%%%%%%%%%%%

\chapter{Preparation}

This section will summarise the work that was done before implementation of any code began. Broadly, this encompasses research on the literature around the topic, learning of new skills to be used during implementation, and planning of the implementation and evaluation portions of the project.


\section{Review of audio fingerprinting techniques}

The first step undertaken was to review the current literature around the area. One particularly useful review was given by Cano et al. \cite{Cano02}, which gives a good overview of the general structure of most audio fingerprinting algorithms. This gave me a good starting point for my system design.

I also focussed on fully understanding the two algorithms which I had considered during my proposal \cite{Haitsma02} \cite{Wang03}. Both use many of the techniques I have learnt in Part II courses, particularly from Digital Signal Processing, which meant I was well placed to implement them well. However, they also have interesting differences in their approaches, and so they were chosen as the two algorithms to be implemented for this project.


\section{Evaluation}

\subsection{Library}

To test the audio fingerprinting algorithms, a reasonably large library of songs was required to match against. Both of the papers I was focussing on tested their algorithms on databases of 10,000 songs, but a library this size would be difficult for me to obtain for this project. A library size of at least 500 songs was decided, as a compromise between time constraints and the quality of the testing.

Genre and style diversity in the song library was also an important consideration, in order to reduce biases in the testing process.

My own local music library was used as the basis for my test set (around 600 songs). Fortunately my library is fairly diverse, but I also supplemented the library with more songs from the Free Music Archive to increase the representation of less common genres in my local collection. The full test library consists of 720 songs.

\subsection{Test types}

Testing the algorithms consists of inputting clips of known songs in the library, and recording whether each clip was matched to the correct song. Different distortions are applied to these clips so as to test the algorithms' resilience to different types of distortions in the input signal.

Three main categories of test clips were chosen, which are summarised below.

\subsubsection{Plain clips}

Plain clips are simply generated by cropping original song audio files to the required length, and are used as a first check during development to make sure the algorithm is functioning at a basic level. The match rate for these should be 100\%.

\subsubsection{Clips with added noise}

Clips with added noise are generated by taking plain clips and overlaying noise at different signal to noise ratios. The noise added might be generated, such as gaussian noise, or taken from a pre-recorded noise file. An example of pre-recorded noise might be a clip of ambient noise in a busy restaurant.

\subsubsection{Re-recorded clips}

Finally, a re-recorded clip is generated by playing a plain clip out of some speakers and re-recording the audio through a microphone. This captures more naturalistic distortions such as echo and reverb, as well as any inaccuracies, frequency biases or clipping introduced by the microphone or speakers.


\section{Software engineering principles}

\subsection{Technology choices}

One of my early focusses was deciding on the technologies to be used for the project. My main considerations when making these decisions were ease of development, ease of evaluation, and my own familiarity.

To make development as easy as possible, I decided to make sure there was strong library support for operations which would be useful but could take a disproportionate amount of time to implement well myself, such as audio file manipulation. For example, a fast and accurate Discrete Fourier Transform implementation was essential.

I investigated several toolkits for music information retrieval, including in particular the C++ library OpenSMILE \cite{Eyben10} and the MATLAB library MIRToolbox \cite{Lartillot07}.

Ease of evaluation was also important. Although there wouldn't be much difference in writing test scripts in most programming languages, some languages, such as MATLAB, provide inbuilt support for plotting graphs, which could speed up my workflow significantly.

I was already somewhat familiar with MATLAB, having written a little for the Part 1A NST Mathematics course, and for various exercises in Part 1B. I was also familiar with C++ through the Part 1B course. However, I would not have described myself as experienced with either language.

On balance, I decided that MATLAB was the better choice, mainly on the merit of its strong inbuilt library support for both graph output and common signal processing operations. The toolkits for the two language appeared to be roughly comparable for my needs.


\subsection{Technology familiarisation}

\subsubsection{MATLAB}

Since I had limited experience using MATLAB, particularly for larger programs, I began by re-familiarising myself with the language. I did this mainly by working through small textbook-style exercises -- for example, implementing bubble sort -- and reading documentation for areas of the language I hadn't previously been exposed to such as namespacing, first-class functions and anonymous functions.

\subsubsection{MIRToolbox}

I also familiarised myself with MIRToolbox and its documentation. I started to investigate how useful it would be by importing some audio from an mp3 file and drawing a spectrogram of it. I implemented this task both with and without MIRToolbox, to compare the perfromance and ease of development. I found that the MIRToolbox version of the code was both slower and harder to develop, in that MIRToolbox tends to constrain you to a particular data flow to optimise certain operations, which makes it less flexible.

After consulting the operations available in the MATLAB standard library, I decided it would probably be possible to implement the algorithms just as easily without the use of MIRToolbox. Since, based on my preliminary tests, the MIRToolbox code might also be slower, I decided not to use MIRToolbox for my implementations.

\subsubsection{Database access}

Early on, I identified database access as a possible difficulty with my algorithm implementations. To ensure this would not be a problem, I decided to finish my familiarisation period by setting up a small database and making sure I could read and write to it.

Although MATLAB has an official Database Toolbox for database access, it is not included under the University's blanket license, so I investigated alternatives before attempting to obtain a license by other means.

I found an open source MATLAB SQLite3 driver \cite{Yamaguchi14} which worked well, and supported the full SQLite feature set, including batching multiple operations into transactions. After reading and writing to a small test database, I decided to use this database system for both my algorithm implementations.


\subsection{Version control}

Git was used as my version control system during development and writeup. This provided a backup of all code via GitHub, as well as the ability to roll back changes and develop new features on separate branches. The commit history was also very useful as a supplement to my written notes when writing up the project.


\subsection{Backup strategy}

Development was carried out on my laptop. Regular full disk backups of the machine were taken using Time Machine onto an external drive, in case of corruption or malfunction. In addition, in case of theft or loss of both the laptop and hard drive, all data was backed up remotely. This was achieved using Google Drive for the music library, and a private GitHub repository for code and documentation.


\subsection{System design}

- Planned structure of code


\subsection{Development strategy}


\section{Starting point}

- Existing stuff used (see starting point in proposal)


%%%%%%%%%%%%%%%%%%%%%%%%%%%%%%%%%%%%%%%%%%%%%%%%%%%%%%%%%%%%%%%

\chapter{Implementation}

- remember to include thoughts about testing at this point

%%%%%%%%%%%%%%%%%%%%%%%%%%%%%%%%%%%%%%%%%%%%%%%%%%%%%%%%%%%%%%%

\chapter{Evaluation}


%%%%%%%%%%%%%%%%%%%%%%%%%%%%%%%%%%%%%%%%%%%%%%%%%%%%%%%%%%%%%%%

\chapter{Conclusion}

In this project I successfully implemented two algorithms for audio fingerprinting and demonstrated their use for song recognition. I gathered a library of songs to test against, and used it to evaluate and compare both algorithms against a variety of criteria.




%%%%%%%%%%%%%%%%%%%%%%%%%%%%%%%%%%%%%%%%%%%%%%%%%%%%%%%%%%%%%%%%%%%%%
% the bibliography
\addcontentsline{toc}{chapter}{Bibliography}
\bibliography{refs}

%%%%%%%%%%%%%%%%%%%%%%%%%%%%%%%%%%%%%%%%%%%%%%%%%%%%%%%%%%%%%%%%%%%%%
% the appendices
\appendix


\chapter{Project Proposal}

\documentclass[12pt]{article}
\usepackage{a4wide}

\newcommand{\al}{$<$}
\newcommand{\ar}{$>$}

\parindent 0pt
\parskip 6pt

\begin{document}



\section*{Introduction and Description of the Work}

Audio fingerprinting algorithms are a way of characterising the features of an audio file and using them to produce a condensed signature which summarises the content of the audio. They are best known for their ability to link unlabelled audio to corresponding metadata, regardless of the format of the audio. This project focuses on song identification against a predetermined library of music, and examines two audio fingerprinting algorithms to investigate their relative strengths and weaknesses.

A source of difficulty for audio fingerprinting algorithms is the potentially high level of variance in audio data which is perceptually very similar. Lossy audio compression formats such as MP3 use this to their advantage by removing many details which are practically inaudible to the human ear. This means that direct comparison of waveforms or cryptographic hash values is not effective. For this reason, audio fingerprinting algorithms must take into account the content of the audio in the context of the human hearing system, instead of simply regarding the audio as digital data to be matched against.

In general, the music identification task has a general form which is reasonably constant between current methods. There are two fundamental processes -- the fingerprinting algorithm, and the matching process. Firstly, a library of music is run through the fingerprinting algorithm to generate a database of fingerprints corresponding to songs. When an audio clip is to be matched, the fingerprinting algorithm is run on the clip, and the resulting fingerprint is compared against the database to find the closest match possible.

The matching process is closely tied to the fingerprinting algorithm, since the method of calculating how similar or different two fingerprints are will vary on how the fingerprint is generated and structured.

Several performance criteria used to assess music identification algorithms include:

\begin{description}

\item [Computational simplicity:] The time taken for both the fingerprinting process, and to take a clip of audio and match it against the database of known songs. Generally what is important is not the exact speed, but the complexity of the matching process as the database grows.

\item [Robustness:] How well the matching process will work in the presence of background noise and other distortions in the input signal, such as compression or reformatting.

\item [Compactness:] The size of each fingerprint. Keeping this small can keep the database compact.

\item [Discrimination power:] The fingerprints need to be distinct enough to give a suitably low probability of false positive matches.

\end{description}

In general, a lot of these criteria are a tradeoff between information loss and dimensionality reduction -- on the one hand, you want a small database, quick matching and robustness to insignificant changes to the input, but on the other hand if you throw away too much information you won't have the discrimination power to distinguish between similar but different audio clips.


\section*{Starting Point}

I have read a number of papers in the area over the summer and am familiar with the general concepts of the two algorithms to be investigated. 

Due to the large proportion of the project which is signal processing, I intend to use MATLAB along with the MIRToolbox library to implement the algorithms.

I am somewhat familiar with MATLAB due to an introduction at the end of the Unix Tools course, and the 3 MATLAB practicals included in the first year Mathematics module in the Natural Sciences Tripos. However, I haven't used it for over a year and a half, and I expect to require some time to re-familiarise myself with the language.

I will also need to learn how to use MIRToolbox, which I am not familiar with. I will also need to learn to do more advanced things such as database access in MATLAB, which I haven't yet done.

A music library will be required to test the algorithms against. I intend to use my own music library of around 700 songs, possibly supplemented with freely distributed music from various internet sites. 

\section*{Substance and Structure of the Project}

The aim of the project is to investigate the performance of several audio fingerprinting and identification algorithms, producing quantitative comparisons of their behaviour for different types of audio clips. 

Although many fingerprinting algorithms have been proposed in the literature, this project will focus on comparing two, with the possibility of implementing others as an extension. 

The two algorithms which will likely be implemented are:

\begin{enumerate}

\item An algorithm which extracts hash values for every few milliseconds by thresholding energy differences of several frequency bands, proposed in \emph{A Highly Robust Audio Fingerprinting System} by Jaap Haitsma and Ton Kalker.

\item An algorithm commercially deployed by the music identification application \emph{Shazam} and presented by Avery Li-Chun Wang in \emph{An Industrial-Strength Audio Search Algorithm}, which finds frequency peaks in the audio and hashes them combinatorially as ``constellations".

\end{enumerate}

The project has the following main sections:

\begin{enumerate}

\item Familiarisation with MATLAB and the surrounding software tools, including MIRToolbox. Detailed research of the current techniques in audio fingerprinting, and in particular the two algorithms to be implemented. 

\item Assembly of a library of music to match against, and a set of clips from that library to test against, with various characteristics such as varying levels of noise and distortion. Distortions should include the effects of mild to severe compression, and background noise tests should include the addition of both random white noise and common background audio such as voices or traffic noise. This background audio can be added to clean clips of audio at different volumes to provide different signal to noise ratios. Suitable background audio clips can be found online free of license restrictions on sites such as the Free Sound Project.

\item Developing and testing the code for each of the matching algorithms.

\item Evaluation of the algorithms: This will involve thorough testing using the library and set of test clips prepared in section 2. The algorithms should be tested on several different criteria such as matching accuracy in the presence of noise or distortions, compactness of the fingerprints and speed of matching against different sized music libraries.

\item Writing the Dissertation.

\end{enumerate}


\section*{Success Criteria}

The following should be achieved:

\begin{itemize}

\item Implement at least two song matching algorithms

\item Assemble and/or create various clips and sets of clips to test the algorithms, as well as a music library to match against

\item Compare the algorithms implemented against several criteria

\end{itemize}


\newpage %%%%%%%%%%%%%%%%%%%%%%%%%%%%%%%%%

\section*{Timetable and Milestones}

\subsection*{Work package 1 (due 23rd Oct)}
Study the core concepts behind audio fingerprinting algorithms. Read up on the two algorithms to be compared to ensure a good understanding before beginning implementation. Familiarisation with MATLAB and MIRToolbox, and any other tools to be used.

Assemble a music library with which to test the algorithms, along with a small set of test clips covering basic test cases. These will be used during development to make sure the implementations are working, at least at a basic level.

\subsection*{Work package 2 (due 7th Dec)}
Design a suitable structure for the implementations, and start work on the code that will be shared between them. For instance, there will be some common functionality to do with reading in audio data and returning matches. Make sure with my supervisor that I have a solid plan for implementation of the algorithms over Christmas.

\emph{Deliverables: Basic structure design for implementations}

\subsection*{Work package 3 (due 9th Jan)}
Implement the two algorithms and finish the shared code over the Christmas break. Test the implementations using the basic test set to make sure they work. Write a first draft of the implementation section of the dissertation.

\emph{Deliverables: Implementations for both algorithms. First draft of implementation section.}

\subsection*{Work package 4 (due 23rd Jan)}
Add to test clip set to build a full set of tests, to test as many criteria as possible.

\emph{Deliverables: Full set of test clips}

\subsection*{Work package 5 (due 30th Jan)}
Write progress report and presentation.

\emph{Deliverables: Progress report and presentation}

\emph{Deadline (30th Jan): Progress report}

\subsection*{Work package 6 (due 1st Mar)}
Fully test the algorithms using the library of tests and produce diagrams to show the results. Write the introduction chapter of the dissertation, and a first draft of the evaluation section. Finish the implementation section.

\emph{Deliverables: Results of thorough tests. Drafts of introduction and evaluation sections. Finished implementation sections.}

\subsection*{Work package 7 (due Mar 15th)}
First draft of dissertation completed (to coincide with end of Lent term, so as to give time for my supervisor to give feedback and for me to concentrate mostly on revision over the Easter break).

\emph{Deliverables: First draft of full dissertation}

\subsection*{Work package 8 (due May 7th)}
Keep the project ticking over, but exam preparation will take precedence at this point. Discuss dissertation with my supervisor, and review the whole project. Make any improvements possible in the time left.

Hand in final draft of dissertation. 

\emph{Deliverables: Final draft of dissertation}

\emph{Deadline (15th May): Dissertation}


\section*{Resources Required}

 MIRToolbox is available freely online, and MATLAB is available under the university license.

I intend to use my own laptop for the project. It is a 13" MacBook Pro with 8GB of RAM and a 2.8GHz Intel Core i5, running OSX 10.9. In the case of its failure, I will use my desktop computer (a Windows desktop with 8GB of RAM and a 3GHz Intel Core i5). I accept full responsibility for both machines and I have made contingency plans to protect myself against hardware and/or software failure.

I will use GitHub for backup and version control of the documents and code associated with my project. I will use Google Drive to back up the music library required for analysis of the algorithms, as well as a local backup stored on an external hard drive using OSX's Time Machine feature.

\end{document}



\end{document}
